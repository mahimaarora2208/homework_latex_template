\documentclass{article}

\usepackage{fancyhdr}
\usepackage{extramarks}
\usepackage{amsmath}
\usepackage{amsthm}
\usepackage{amsfonts}
\usepackage{tikz}
\usepackage{algorithm}
\usepackage{algpseudocode}

\usetikzlibrary{automata,positioning}

%
% Basic Document Settings
%

\topmargin=-0.45in
\evensidemargin=0in
\oddsidemargin=0in
\textwidth=6.5in
\textheight=9.0in
\headsep=0.25in

\linespread{1.1}

\pagestyle{fancy}
\lhead{\hmwkAuthorName}
\chead{\hmwkClass\ : \hmwkTitle}
\rhead{\firstxmark}
\lfoot{\lastxmark}
\cfoot{\thepage}

\renewcommand\headrulewidth{0.4pt}
\renewcommand\footrulewidth{0.4pt}

\setlength\parindent{0pt}

%
% Create Problem Sections
%

\newcommand{\enterProblemHeader}[1]{
    \nobreak\extramarks{}{Problem \arabic{#1} continued on next page\ldots}\nobreak{}
    \nobreak\extramarks{Problem \arabic{#1} (continued)}{Problem \arabic{#1} continued on next page\ldots}\nobreak{}
}

\newcommand{\exitProblemHeader}[1]{
    \nobreak\extramarks{Problem \arabic{#1} (continued)}{Problem \arabic{#1} continued on next page\ldots}\nobreak{}
    \stepcounter{#1}
    \nobreak\extramarks{Problem \arabic{#1}}{}\nobreak{}
}

\setcounter{secnumdepth}{0}
\newcounter{partCounter}
\newcounter{homeworkProblemCounter}
\setcounter{homeworkProblemCounter}{1}
\nobreak\extramarks{Problem \arabic{homeworkProblemCounter}}{}\nobreak{}

%
% Homework Problem Environment
%
% This environment takes an optional argument. When given, it will adjust the
% problem counter. This is useful for when the problems given for your
% assignment aren't sequential. See the last 3 problems of this template for an
% example.
%
\newenvironment{homeworkProblem}[1][-1]{
    \ifnum#1>0
        \setcounter{homeworkProblemCounter}{#1}
    \fi
    \section{Problem \arabic{homeworkProblemCounter}}
    \setcounter{partCounter}{1}
    \enterProblemHeader{homeworkProblemCounter}
}{
    \exitProblemHeader{homeworkProblemCounter}
}

%
% Homework Details
%   - Title
%   - Due date
%   - Class
%   - Section/Time
%   - Instructor
%   - Author
%

\newcommand{\hmwkTitle}{Homework\ \#1}
\newcommand{\hmwkDueDate}{February 09, 2023}
\newcommand{\hmwkClass}{ENPM809X: Data and Algorithms}
\newcommand{\hmwkClassTime}{Section 0101}
\newcommand{\hmwkAuthorName}{\textbf{Mahima Arora}}

%
% Title Page
%

\title{
    \vspace{2in}
    \textmd{\textbf{\hmwkClass:\ \hmwkTitle}}\\
    \normalsize\vspace{0.1in}\small{Due\ on\ \hmwkDueDate\ at 4:00pm}\\
    \vspace{0.1in}\large{\textit{\hmwkClassTime}}
    \vspace{3in}
}

\author{\hmwkAuthorName}
\date{}

\renewcommand{\part}[1]{\textbf{\large Part \Alph{partCounter}}\stepcounter{partCounter}\\}

%
% Various Helper Commands
%

% Useful for algorithms
\newcommand{\alg}[1]{\textsc{\bfseries \footnotesize #1}}

% For derivatives
\newcommand{\deriv}[1]{\frac{\mathrm{d}}{\mathrm{d}x} (#1)}

% For partial derivatives
\newcommand{\pderiv}[2]{\frac{\partial}{\partial #1} (#2)}

% Integral dx
\newcommand{\dx}{\mathrm{d}x}

% Alias for the Solution section header
\newcommand{\solution}{\textbf{\large Solution}}

% Probability commands: Expectation, Variance, Covariance, Bias
\newcommand{\E}{\mathrm{E}}
\newcommand{\Var}{\mathrm{Var}}
\newcommand{\Cov}{\mathrm{Cov}}
\newcommand{\Bias}{\mathrm{Bias}}

\begin{document}
\maketitle
\pagebreak

% BEGIN WRITING PROBLEM HERE
\begin{homeworkProblem}
\textbf{2.2(b-d)} Bubble Sort and related three parts(refer to HW1 for full questions). \\

\solution \\

\textbf{PART B}

Loop invariant for loop in lines 2-4 is as follows: \\
At the start of each iteration, the sub-array A[j....n] consists of elements originally in that sub-array before entering the loop but in a different order and the first element is the smallest among them.
\begin{enumerate}
    \item \textbf{Verify initialization}: Initially, array A[1] is the single original element and it is sorted.
    \item \textbf{Verify iteration}: In each iteration, A[j] is compared with A[j-1] making A[j-1] the smallest in the sub-array. After the jth iteration, the largest j elements are moved to the end of the array making A.size() - i - 1 element sorted. 
    \item \textbf{Verify termination}: This loop terminates when j = i + 1 and the sorted array is A[1...n] which is the entire input.
\end{enumerate}

\textbf{PART C}

Loop invariant for loop in lines 1-4 is as follows: \\
At the start of each iteration for the first for loop, the sub-array A[1....i-1] consists of elements that are smaller than elements in sub-array A[i...n] in sorted order(first i-1 numbers will be sorted after i iterations).
\begin{enumerate}
    \item \textbf{Verify initialization}: Initially, array A[1...i-1] is empty and hence, the smallest element of the sub-array.
    \item \textbf{Verify iteration}: After execution of the inner loop, A[i] will be smallest of the sub-array A[i....n] and in the start of the outer loop, A[1....i-1] sub-array consists of elements smaller than elements of A[i....n] in sorted order.
    \item \textbf{Verify termination}: This loop terminates when i = A.length - 1 and the sorted array is A[1...n] which is the entire input.
\end{enumerate}

\textbf{PART D} 
The Worst-case running time of Bubble Sort is $\Theta(n^2)$ if the array is reversed. Insertion Sort has a running time that varies from $\Theta(n)$ to $\Theta(n^2)$. Comparing both, Insertion Sort is slightly better than bubble sort as insertion sort has fewer swaps as compared to bubble sort.
\end{homeworkProblem}

\pagebreak

\begin{homeworkProblem}
\textbf{2.3-4(a-c)} Recursive process for Insertion Sort including pseudocode, running time, and solving recurrence to find the complexity(refer to HW1 for full questions).

\solution \\

\part

Pseudocode implementation of the recursive algorithm: \\
\alg{Insertion-Sort-Recursion($Array,n$)}
    \begin{algorithm}[]
        \begin{algorithmic}
            \Function{Insertion-Sort-Recursion}{$Array, n$}
                \If{ $n \geq 1$}
                    \State \Call{Insertion-Sort-Recursion}{$Array, n-1$}
                    \State \Call{Substitute}{$Array, n$}
                 \EndIf
            \EndFunction
            \Function{Substitute}{$Array, m$}
                \State $key \gets Array[m]$
                \State $idx \gets m-1$
                \While{ $idx > 1 AND Array[idx]>key$}
                    \State $Array[idx+1] \gets Array[idx]$
                    \State $idx \gets idx - 1$
                \EndWhile
                \State $Array[idx+1] \gets Array[idx]$  
            \EndFunction{}
        \end{algorithmic}
        \caption{Recursive Insertion Sort}
    \end{algorithm}

 \part
 
 Recurrence for the running time of the recursive insertion sort: \\
 
 \begin{equation}
  T(n)=\begin{cases}
    \Theta(1), & \text{if $n=1$}.\\
    T(n-1) + \Theta(n), & \text{if $n > 1$}.
  \end{cases}
\end{equation}
 
\part

Solving the recurrence, time complexity comes out to be: $\Theta(n^2)$. Let $\Theta(n) = n$. Substituting this value in the equation above, we get \\

   \[
        \begin{split} 
           T(n) = T(n-1) + n \\
           T(n-1) = T(n-2) + n - 1 \\
           T(n) = T(n-2) + 2n -1 \\
           \vdots \\
           T(n) = T(n-k) + kn - \dfrac{(k-1)k}{2}
        \end{split}
    \]
    Setting k = n - 1,
    \[
        \begin{split} 
           T(n) = T(n-n+1) + (n-1)n -\dfrac{(n-1-1)(n-1)}{2} \\
           T(n) = T(1) + (n-1)n -\dfrac{(n-2)(n-1)}{2} \\
           T(n) = T(1) + \dfrac{(n+2)(n-1)}{2} \\
           T(n) = \mathcal{O}(n^2)
        \end{split}
    \]

 
\end{homeworkProblem}

\begin{homeworkProblem}
\textbf{4.1-5} Use the following ideas to develop a non-recursive, linear-time algorithm for the maximum sub-array problem: Start at the left end of the array, and progress toward the right, keeping track of the maximum sub-array seen so far. Knowing a maximum sub-array of A[1....j], extend the answer to find a maximum sub-array ending at index j+1 by using the following observation: A maximum sub-array of A[1.. j+1] is either a maximum sub-array of A[1.. j] or a sub-array A[i.. j+1], for some $1 \leq i \leq j + 1$. Determine a maximum sub-array of the form A[i.. j+1] in constant time based on knowing a maximum sub-array ending at index j.
\\

\solution \\
We can use Kadane's Algorithm to get a linear running time. The algorithm is as follows: \\

\alg{find-max-subarray($A$)}
    \begin{algorithm}[]
        \begin{algorithmic}[1]
            \Function{find-max-subarray}{$A$}
                \State $maxSum \gets nums[0]$
                \State $i \gets 1$
                \While {$i < A.length$}
                \State $nums[i] \gets max(nums[i], nums[i]+nums[i-1])$
                \State $maxSum \gets max(nums[i], maxSum)$
                \EndWhile \\
                \Return $\textit{maxSum}$
            \EndFunction{}
        \end{algorithmic}
        \caption{Kadane's Algorithm for Linear Time}
    \end{algorithm}

This algorithm returns the max sub-array for a given array in Linear time. The max sum is stored in the variable called maxSum and after each iteration, it updates the [i-1] value with the latest sum. It keeps comparing this [i-1] value with the maxSum and stores the greater value in the same variable, hence, always keeping track of max-subarray in A[1.....j].
\end{homeworkProblem}

\pagebreak

\begin{homeworkProblem}
\textbf{4.5-1} Use the master method to give tight asymptotic bounds for the following recurrences.
\textbf{Solution} \\

\part

\(T(n) = 2T(n/4)+1\) 

 \textbf{The answer is $\Theta(\sqrt{n})$.} \\

Since $f(n) = 1$ and $T(n) = n^{\log_b a}$ where a = 2 and b = 4. Therefore,
    \[
        \begin{split} 
           f(n) = n^{\log_4 2-1/2} = 1 \\
           T(n) = n^{\log_4 2} = \sqrt{n}
        \end{split}
    \]
    Thus Case I of the Master Method applies where T(n) is more dominant than f(n).
    \\

\part

\(T(n) = 2T(n/4)+ \sqrt{n}\)

 \textbf{The answer is $\Theta(\sqrt{n}\log n)$.} \\

Since $f(n) = \sqrt{n}$ and $T(n) = n^{\log_b a}$ where a = 2 and b = 4. Therefore,
    \[
        \begin{split}
           f(n) = n^{\log_4 2} = \sqrt{n} \\
           T(n) = n^{\log_4 2} = \sqrt{n}
        \end{split}
    \]
 Thus Case II of the Master Method applies where T(n) equals f(n). 
    \\

\part

\(T(n) = 2T(n/4)+n\)

\textbf{The answer is $\Theta(n)$.} \\

Since $f(n) = n$ and $T(n) = n^{\log_b a}$ where a = 2 and b = 4. Therefore,
    \[
        \begin{split}
           f(n) = n^{\log_4 2 +1/2} = n \\
           T(n) = n^{\log_4 2} = \sqrt{n}
        \end{split}
    \]
 Thus Case III of the Master Method applies where f(n) is more dominant than T(n). 
    \\
  
\part

\(T(n) = 2T(n/4)+n^2\)

\textbf{The answer is $\Theta(n^2)$.} \\

Since $f(n) = n^2$ and $T(n) = n^{\log_b a}$ where a = 2 and b = 4. Therefore,
    \[
        \begin{split}
           f(n) = n^{\log_4 2 +3/2} = n^2 \\
           T(n) = n^{\log_4 2} = \sqrt{n}
        \end{split}
    \]
 Thus Case III of the Master Method applies where f(n) is more dominant than T(n). 
    \\

\end{homeworkProblem}

\end{document}